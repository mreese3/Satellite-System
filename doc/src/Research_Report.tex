\documentclass{article}
\usepackage[letterpaper, margin=1in]{geometry}
\usepackage[en-US]{datetime2}
\usepackage{multicol}
\usepackage[blocks]{authblk}
\usepackage{lipsum}

%Hacky ruler definition for use in checking centering.
%Will be removed later.
\def\bars#1{\hbox to #1{\vrule width 0pt height 1mm depth 2mm 
	\vrule\morebars\morebars}}
\def\morebars{\hfil\vrule\hfil\vrule\hfil\vrule\hfil\vrule\hfil\vrule}
\def\ruler#1{\vbox{\bars{#1}\hrule}}

%Special Latex hackery to allow for a thin Abstract, without losing centering
%or having to force multiple environment. This basically makes abstract
%a quotation environment, with variable left margin.
%Pulled from https://tex.stackexchange.com/questions/151583
%Note that this is hilariously unportable, permanently futzes with the
%definition of abstract.
\let\oldabstract\abstract
\let\oldendabstract\endabstract
\makeatletter
\renewenvironment{abstract} {
	%Changing the leftmargin call 
	\renewenvironment{quotation} {
		\list{}{
			%Set the left margin to adjust the width of the
			%abstract. It's not ideal, but it's consistent.
			\addtolength{\leftmargin}{.40in}
			\listparindent 1.5em%
			\itemindent
			\listparindent
			\rightmargin
			\leftmargin
			\parsep \z@\@plus\p@
			}
		\item\relax
		}
	{\endlist}
	\oldabstract
	}
{\oldendabstract}
\makeatother

\DTMsavedate{creationdate}{2016-03-11}

\title{
	Portable Data Server and Client-side Data Parser/Aggregator for
	Arbitrary Remote Stations
	}
\date{\DTMusedate{creationdate}}
\author[]{Sean Tracy}
\author[]{James Wagner}
\author[]{Mike Reese}
\author[]{\authorcr Faculty Advisor: Dr. Haklim Kimm}
\affil[]{}


\begin{document}
	\maketitle
	\vfill
	\begin{abstract}
		\large{\noindent
			A multitude of radio devices exist for the transmission
			of data from small amateur and educational satellites.
			The simplest radio modules do not always support
			data-integrity features like those found in the
			Transport Layer of the TCP/IP stack, and such systems
			may be limited in practicality to only broadcasting a
			FM or CW beacon. If the radio module does support some
			form of data-integrity protection, its use may be
			dependent on continued bidirectional communication with
			the receiving station. Such an arrangement is likely to
			be subject to loss due to a weak signal on one or both
			sides of the link, and is far from ideal for
			organizations lacking in reception capability. This
			project is an attempt at the design and implementation
			of a lightweight and portable system for data
			transmission and logging without the overhead of
			bidirectional communication or the availability of a
			high bandwidth link, for use on affordable SoC
			computers like the Raspberry Pi.
			}
		\end{abstract}
	\vfill
	\clearpage

	\begin{multicols}{2}
		\section{Background}
		\lipsum[1-3]
		\section{Implementation}
		\lipsum[4-6]
		\section{Testing and Refinement}
		\lipsum[7-9]
		\section{Results}
		\lipsum[10-12]
	\end{multicols}

\end{document}
